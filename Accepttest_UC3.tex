%\chapter{Accepttest Specifikation}
\section{Test af Use-case 3}
\begin{table}[H]
	\centering
	\caption{Accepttestspecifikation UC3 Hovedscenarie}
	\label{ATUC1:Hovedscenarie}
	%Se Tabel \ref{ATUC2:Hovedscenarie} på side \pageref{ATUC2:Hovedscenarie}
	\begin{tabular}{ p{80pt}  p{320pt} }\hline
		\rowcolor{white}	
		\textbf{Use-case under test} & U3: Indstil tid \\
		\rowcolor{lightgray}
		\textbf{Scenarie} & Hovedscenarie \\\rowcolor{white}	
		\textbf{Prækondition} & Bruger befinder sig i hovedmenu.
 \\
		\hline
	\end{tabular}
	\begin{tabular}{  p{26pt} p{100pt}  p{101pt} | p{67pt} | p{68pt}}
		\textbf{Step} & \textbf{Handling} & \textbf{Forventet observation/resultat} & \textbf{Faktisk observation/resultat} & \textbf{Vurdering (OK/FAIL)}\\
		1 & Tryk på knappen \emph{Indstillinger} på brugergrænsefladen & Brugergrænsefladen skifter menu til indstillingsmenuen &  &  \\
		2 & Tryk på tekstboksen med timeantal & En dropdownmenu med antal timer fra 0 til 23 dukker op.  &  & \\
		3 & Tryk på timeantallet 12 & Dropdownmenuen forsvinder og det valgte timeantal står nu i tekstboksen, og bekræftelsesknappen bliver synlig. &  &  \\
		
		4 & Ændre minutter til 30 & Bekræftelsesknappen dukker op &  &  \\	
		5 & Tryk på bekræftelsesknappen & Tiden er nu skiftet til 12:30.  &  &  \\
		\hline
	\end{tabular}
\end{table}

\begin{table}[H]
	\centering
	\caption{Accepttestspecifikation UC3 Ext. 1: Bruger trykker på tilbageknap}
	\label{ATUC1:Ext1}
	%Se Tabel \ref{ATUC2:Hovedscenarie} på side \pageref{ATUC2:Hovedscenarie}
	\begin{tabular}{ p{80pt}  p{320pt} }\hline
		\rowcolor{lightgray}	
		\textbf{Use-case under test} & U3: Indstil tid \\
		\rowcolor{white}
		\textbf{Scenarie} & Ext. 1: Bruger trykker på tilbageknap \\\rowcolor{lightgray}	
		\textbf{Prækondition} &
		Bruger er inde i indstillingsmenuen \\
		\hline
	\end{tabular}
	\begin{tabular}{  p{26pt} p{100pt}  p{101pt} | p{67pt} | p{68pt}}
		\textbf{Step} & \textbf{Handling} & \textbf{Forventet observation/resultat} & \textbf{Faktisk observation/resultat} & \textbf{Vurdering (OK/FAIL)}\\
		1 & Tryk på tilbageknap & Brugergrænseflade er tilbage i hovedemenu
 &  &  \\
		\hline
	\end{tabular}
\end{table}

